\documentclass[12pt]{article}

\usepackage[utf8]{inputenc}

\usepackage{graphicx}
\graphicspath{ {./images/} }
\usepackage{csquotes}
\usepackage{geometry}
\usepackage[colorlinks]{hyperref}
\hypersetup{
  linkcolor=blue,
  anchorcolor=,
  citecolor=,
  filecolor=,
  menucolor=,
  runcolor=,
  urlcolor=blue,
}
\geometry{a4paper, margin=3cm}
\font\titlefont=cmr12 at 18pt
\setlength{\parindent}{0em}

\title{\titlefont The Settlers II (10th Anniversary) - An advanced guide}
\author{TheCheese}
\date{}

\begin{document}

\maketitle

\vspace{2cm}

\begingroup
  \hypersetup{hidelinks}
  \tableofcontents
\endgroup


\section{Introduction}
\label{sec:introduction}

\begin{displayquote}
\textbf{The Settlers II (10th Anniversary)} (German: Die Siedler II: Die nächste Generation) is a city-building game with real-time strategy elements, developed by Blue Byte and published by Ubisoft. Released for Microsoft Windows in September 2006, it is a remake of The Settlers II (1996).
\end{displayquote}
\hspace{2cm}- \textit{\href{https://en.wikipedia.org/wiki/The_Settlers_II_(10th_Anniversary)}{Wikipedia}}\\

This guide is not just about getting started with the game, as well as some tricks and tips, but also aims to cover more advanced techniques and strategies, along with specialized optimizations and common pitfalls.

Have fun reading!

\subsection{Contributing}
\label{sec:contributing}

This guide's \LaTeX-code is available on GitHub. If you want to add something, found out something new or just want to report false content, please submit a pull request or open an issue: \url{https://github.com/TheCheese42/ts2-anniversary-guide}.

\section{Getting Started}
\label{sec:gettingstarted}

\subsection{Wood industry}
\label{sec:woodindustry}

The first thing you'll need is enough wood to expand using barracks and to build Mines, together with a solid food industry.

The \hyperref[sec:woodcutter]{Woodcutter} and the \hyperref[sec:forester]{Forester} both produce about 60 items per hour, while the \hyperref[sec:sawmill]{Sawmill} produces 80 items per hour. Therefore, it's best to use 4 \hyperref[sec:woodcutter]{Woodcutters} and \hyperref[sec:forester]{Foresters} with 3 \hyperref[sec:sawmill]{Sawmills}. If you need more, add 1 to every of those, however this setup is very sufficient most of the time, unless you need to make a big fleet of \hyperref[sec:ship]{Ships} right at the beginning.

\subsection{Expanding}
\label{sec:expanding}

To have more space for your industry and to reach more mining spots, as well as pushing the enemy, you will need to expand your territory as quickly as possible. \hyperref[sec:barrack]{Barracks} are the most efficient buildings to do so, as they only require 2 planks and therefore a minimum of 60 seconds to build. You should expand in every possible direction to quickly gain a size advantage over your enemies. To speed up the process, you can use some advanced mechanics, like using \hyperref[sec:ghostbuildings]{ghost buildings}.

\subsection{Mines and Food}
\label{sec:minesandfoot}

The first mining resource you'll need is gold. If you are the first one to find and use gold, you can quickly overcome the nearest enemy. To upgrade your soldiers, you need a \hyperref[sec:mint]{Mint}, that requires one gold and one coal per coin. The production speed of a \hyperref[sec:mine]{Mine} and a \hyperref[sec:mint]{Mint} is in both cases 49 seconds per item, so you need exactly one \hyperref[sec:coalmine]{Coal-} and one \hyperref[sec:goldmine]{Gold Mine} to keep a \hyperref[sec:mint]{Mint} running. Most of the time, 2 \hyperref[sec:mint]{Mints} are enough for a run, so you'll want to not have more than 2 \hyperref[sec:goldmine]{Gold Mines} running at a time. Make sure you spread those as much as possible to avoid having to destroy them again early.

Depending on your \hyperref[sec:startresources]{starting resources}, you might need to make new tools. As you already have between 8 and 32 iron in your headquarter, you should use that for your \hyperref[sec:toolmaker]{Toolmaker} and turn his production off after the materials are used up, assuming you don't need any further tools.

The next type of \hyperref[sec:mine]{Mine} you will need is an \hyperref[sec:ironmine]{Iron Mine} to make new soldiers. A \hyperref[sec:smithy]{Smithy} required one iron and one Coal per weapon. To get iron, you also demand a \hyperref[sec:smeltery]{Smeltery}, which requires one iron ore and one coal ore per iron. As both buildings take 49 seconds to fulfill their work, you need one \hyperref[sec:ironmine]{Iron Mine} and two \hyperref[sec:coalmine]{Coal Mines} to keep a \hyperref[sec:smeltery]{Smeltery} + \hyperref[sec:smithy]{Smithy} setup running. However, it's often advisable to have \textbf{two} \hyperref[sec:smithy]{Smithies} and one \hyperref[sec:brewery]{Brewery} running, so you should double that number.

Lastly, there are \hyperref[sec:stonemine]{Stone Mines}. If you are \textbf{not} playing on low \hyperref[sec:startresources]{starting resources}, or you've got a stonecutter working, you won't need those at first. Place them as you need them.

Assuming you didn't build any \hyperref[sec:stonemine]{Stone Mines} immediately, we are at 2 \hyperref[sec:goldmine]{Gold Mines}, 2 \hyperref[sec:ironmine]{Iron Mines} and 6 \hyperref[sec:coalmine]{Coal Mines}, making 10 in total. To get all those miners working, a good \hyperref[sec:foodsystems]{food system} is essential. Also, check out this \hyperref[sec:foodsystemcomparison]{food system comparison} to decide what suit's best for your particular map.

In conclusion, start by building some \hyperref[sec:fisherman]{Fisherman} and \hyperref[sec:hunter]{Hunter}, as long as it makes sense and you have enough tools.
Then continue investing in \hyperref[sec:farm]{Farms}, \hyperref[sec:mill]{Mills} and \hyperref[sec:bakery]{Bakeries} for a healthy, vegan nutrition for your fellow miners.

\subsection{Decentralization}
\label{sec:decentralization}

To avoid long transport ways you should refine your products right where they are being produced. Therefore you should build a \hyperref[sec:storehouse]{Storehouse} right next to your \hyperref[sec:farm]{Farms} and \hyperref[sec:mine]{Mines}. To get your industry running decentralized and your expansion going on quickly, you have to get some wood and, if possible, stone to those \hyperref[sec:storehouse]{Storehouses} as well. For wood it's enough to build one \hyperref[sec:woodcutter]{Woodcutter}, one \hyperref[sec:forester]{Forester} and one \hyperref[sec:sawmill]{Sawmill} next to the \hyperref[sec:storehouse]{Storehouse}. For stone however, you gotta be lucky, that there is some stone lying around.

Common problems in the middle of a game are traffic jams around your headquarter. To prevent those, it's a good idea to not build your \hyperref[sec:metalworks]{Metalworks}, \hyperref[sec:smithy]{Smithies}, \hyperref[sec:smeltery]{Smelteries}, \hyperref[sec:mint]{Mints}, \hyperref[sec:mill]{Mills}, \hyperref[sec:bakery]{Bakeries}, etc. near the headquarter. Especially the iron- and gold-processors should be near to where you actually need them, which is the enemies border.

\subsection{First Contact}
\label{sec:firstcontact}

Depending on the map, mission and skill set of the participants, you will encounter your enemies earlier or later. In any case, you will want to have the better military strength. To boost your likelihood of winning fights  early in the game, it's important to have a solid gold industry running. This ideally consists of 2 \hyperref[sec:goldmine]{Gold Mines}, 2 \hyperref[sec:coalmine]{Coal Mines} and 2 \hyperref[sec:mint]{Mints}, unless you are facing many enemies at the same time. If you cannot find any gold, you'll want to go for strength in numbers by training a whole lot of \hyperref[sec:private]{Privates}. However, before attacking an enemy, you should \textbf{always} check the \hyperref[sec:statisticsmilitary]{Military Strength statistic}, as well as the \hyperref[sec:statisticscoins]{Coins Produced statistic}. Those will tell you whether or not you will stand a chance. Anyways, if you've got \hyperref[sec:general]{Generals} while your enemy did not produce any coins yet (or just started doing so), attack as long as you have the advantage. If you see any gold infrastructure like \hyperref[sec:goldmine]{Mines}, \hyperref[sec:mint]{Mints} or even just a \hyperref[sec:storehouse]{Storehouse}, that's what you should be going for in the first place.

\section{Military}
\label{sec:military}

\subsection{Soldier Industry}

To fight you need to recruit and upgrade your soldiers. For this, there are two main approaches. The first one is pretty simple and straightforward, while the second one is very powerful and prevents you from sending \hyperref[sec:privates]{Privates} (or other low-level soldiers) into fight.

\subsubsection{General Farm}
\label{sec:generalfarm}

To recruit a single \hyperref[sec:private]{Private}, \hyperref[sec:smithy]{one Sword, one Shield} and one \hyperref[sec:brewery]{Beer} need to be in a single \hyperref[sec:storehouse]{Storehouse}, a private will spawn automatically. To level up Soldiers, all you need is a \hyperref[sec:mint]{Coin} in a military building. One Soldier of every level gets promoted by one (i.e., if you have one of each \hyperref[sec:private]{Private}, \hyperref[sec:corporal]{Corporal}, \hyperref[sec:sergeant]{Sergeant} and \hyperref[sec:officer]{Officer}, all of them get promoted.

Of course you could just let the coins be delivered directly to the front. However, it's much more efficient if you only allow coins in one building near the Headquarter, so you can let \hyperref[sec:private]{Privates} come in and let them promote, then throw them out again. For this to work, you need to prefer weak defensive Soldiers in you global economy settings.

\subsubsection{General Generator}
\label{sec:generalgenerator}

As you might have noticed, the previous system has some drawbacks. Mainly you need to modify the global economy settings badly, so that there will be only \hyperref[sec:private]{Privates} in your military buildings, unless you manage to train everyone to the maximum level. To avoid this, you can build a setup where everything required to recruit and promote a Soldier gets brought into a single \hyperref[sec:storehouse]{Storehouse}. For promotion, you should build a \hyperref[sec:watchtower]{Watchtower} or, preferably, a \hyperref[sec:stronghold]{Stronghold} connected to the \hyperref[sec:storehouse]{Storehouse}. To prevent \hyperref[sec:private]{Privates} from coming out, There shouldn't be a road to the rest of your civilization, so to transport goods but not people, you have to use a water road. Now, go to your \hyperref[sec:storehouse]{Storehouse} and block any Soldiers but \hyperref[sec:private]{Privates} from going in.

In the end, if you've got \hyperref[sec:brewery]{Beer}, \hyperref[sec:smithy]{Swords and Shields} in that exact \hyperref[sec:storehouse]{Storehouse}, a \hyperref[sec:private]{Private} will spawn, go to the connected \hyperref[sec:watchtower]{Watchtower} or \hyperref[sec:stronghold]{Stronghold} and upgrade to \hyperref[sec:general]{General} level. As one Soldier (if you want to prefer strong soldiers in your global economy settings, even though it shouldn't be needed when using this technique, you must wait until all soldiers got promoted. However, this is very inefficient as one Soldier of every level could get promoted otherwise) in the building becomes a \hyperref[sec:general]{General}, you can evacuate him so there's space for another \hyperref[sec:private]{Private}. The evacuated \hyperref[sec:general]{General} can't go back to the \hyperref[sec:storehouse]{Storehouse}, so he will search for the nearest road and go to a different \hyperref[sec:storehouse]{Storehouse}, \hyperref[sec:harbor]{Harbor}, or your Headquarter.

\vspace{0.5cm}
\includegraphics[width=\textwidth]{generalgenerator_crop}
\vspace{0.5cm}

This method should forbid non-\hyperref[sec:general]{Generals} from fighting at all, while making Soldier promotion very quick.

\subsection{Military Tricks}
\label{sec:militarytricks}

The following section is a collection of useful tricks regarding military fights. Not only can you use them to your advantage, but you should also learn how to deal with them when applied by your opponents. Most of the concepts have crucial drawbacks that you must be aware of.

\subsubsection{Using weakened enemies as rampart}
\label{sec:militaryrampart}

When fighting against multiple enemies, situation can come up, where you just defeated a tribe and immediately get to face the next one. This can be very dangerous, as you are defending on new territory without much industry, while your enemy could be on familiar grounds. Additionally, you likely lost a bunch of soldiers in previous fights. Preventing this situation should be your top priority.

A possible technique is to leave a small corridor of the almost destroyed enemy along your realm's boundary. This way, the new enemy would have to attack through the weakened enemy to reach you. Note that this is most effective if the two enemy folks are in the same team.

<insert image here>

If they're not in the same team though, you can still implement this technique by making a corridor a bit wider. Still you have to get your \hyperref[sec:decentralization]{decentralized industry} up as fast as possible, before your future opponent takes the remaining territory.

\subsubsection{Destroying Buildings}
\label{sec:destroyingbuildings}

In this game you are \textbf{never} guaranteed to have greater military force than your opponents. That's why it's essential to have good backup strategies in case of you being overwhelmed. One such strategy is destroying military buildings before the enemy can take them.

Whenever you see one of your military building attacked with not many chances of survival, you need to check the cause of your loss. Examples would be a poorly distributed military (e.g. strong soldiers aren't where they're needed the most; buildings far from enemy territory are occupied, and so on). This needs to be resolved quickly, however, retaking the building should be doable. However, in situations when your general \hyperref[sec:statisticsmilitary]{military strength} is far worse than your enemy's, the newly captured building won't be retakeable and will make it easy to reach further into your domain, without you having any chance to recover, almost certainly sealing your doom.

\label{wastingenemyresourcesbydestroyingbuildings}
If you manage to destroy your buildings before losing the fight, you cause the enemy to waste costly time, as they need to create new buildings, likely at least a \hyperref[sec:watchtower]{Watchtower}, which takes quite long to build while also consuming valuable resources. To make even more out of this situation, read about \hyperref[sec:buildingcanceling]{canceling buildings}.

However, destroying military buildings is not always the best choice. Even if you have far less soldiers, there is a chance that you have some \hyperref[sec:general]{Generals} while your opponent only got \hyperref[sec:private]{Privates}. In this situation it may be smarter to let your troops fight to death, as they will kill far more Privates than your enemy can spare. Retaking the building can wait until some fresh soldiers are upgraded.

\subsubsection{Building Canceling}
\label{sec:buildingcanceling}

In the case of an enemy is arming next to your border, either because you just met for the first time, or the \hyperref[wastingenemyresourcesbydestroyingbuildings]{scenario} described in the \hyperref[sec:destroyingbuildings]{Destroying Buildings} section took place and you don't feel ready for contact, you can delay attacks as long as your shared border is quite narrow.

To pull this off, let your opponent build a military building next to your border. This is gonna be a \hyperref[sec:watchtower]{Watchtower} or \hyperref[sec:stronghold]{Stronghold}, most of the time. Those building take a long time to build, so you should have enough time to build a \hyperref[sec:barrack]{Barrack} on your side of the border, which should clearly be finished first. This will destroy your opponent's construction side, letting time and materials go to waste. To guarrantee being done first, even if the enemy's building is just a \hyperref[sec:guardhouse]{Guard House}, you can get a local \hyperref[sec:woodindustry]{wood industry} up, together with a \hyperref[sec:storehouse]{storehouse} containing soldiers.

After canceling the building you should destroy the barrack again to give your enemy room for another building. Especially bots won't learn out of this trick.

\subsubsection{Complete Erasure}
\label{sec:completeerasure}

No, this is no suicide advice. In some situations there just isn't another option than forcing your opponent to approach you slowly by creating new buildings. To accomplish this, you can first silently evacuate important goods from storehouses near the front and then destroy all military buildings in a medium to big distance to the border at once. This way you can try to recover your industry (see the \hyperref[sec:minesandfoot]{Mines and Food} and \hyperref[sec:decentralization]{Decentralization} sections) and possibly overwhelm your enemy due to hurried \hyperref[sec:overexpansion]{overexpanding}.

Note however, that this can greatly damage your economy if important mines are destroyed in the process. Make sure you got alternatives running \textbf{before} erasing.

\section{Food Systems}
\label{sec:foodsystems}

There are mainly 4 types of food:

\begin{itemize}
  \item Fish
  \item Bread
  \item Ham (\hyperref[sec:hunter]{Hunter})
  \item Ham (\hyperref[sec:slaughterhouse]{Slaughter})
\end{itemize}

\subsection{Fish}
\label{sec:fish}

Fishing is a great way to get food in the early game. One \hyperref[sec:fisherman]{Fisherman} produces \textbf{roughly} one fish every 190 seconds. Therefore you need 4 \hyperref[sec:fisherman]{Fisherman} to keep a single mine running. Depending on your \hyperref[sec:startresources]{starting resources}, you will have a total of 3, 6 or 12 fishers available right at the beginning. Make sure you don't exceed this limit, as investing metal to employ more fisherman is just not worth it when comparing to other food systems.

Note, that fishing is non-renewable.

\subsection{Bread}
\label{sec:bread}

Bread is the best source of food during the game. It's renewable and you only need one \hyperref[sec:bakery]{Bakery} to keep a \hyperref[sec:mine]{Mine} running, as both buildings take 49 seconds to produce a good. To keep a \hyperref[sec:bakery]{Bakery} running you need one \hyperref[sec:mill]{Mill}, to keep that running you need about 1.75 \hyperref[sec:farm]{Farms}. The \hyperref[sec:bakery]{Bakery} also demands one bucket of water. Luckily, the \hyperref[sec:well]{Well Worker} also takes exactly 49 seconds to work. That being said, a great ratio for a bread system is 7 \hyperref[sec:farm]{Farms}, 4 \hyperref[sec:mill]{Mills}, 4 \hyperref[sec:bakery]{Bakeries} and 4 \hyperref[sec:well]{Wells}.

This system can feed 4 \hyperref[sec:mine]{Mines} for just 11 tools in total.

\subsection{Ham (Hunter)}
\label{sec:hamhunter}

The  \hyperref[sec:hunter]{Hunter} can roughly produce one ham every 70 seconds under optimal conditions. While that is about thrice as fast as the fisherman, you usually don't get more than 20 ham for a big area, as animals don't renew at all. Hunters can be great for gaining some food very early in the game, however they're not worth considering anymore as you invest in other food systems.

\subsection{Ham (Slaughter)}
\label{sec:hamslaughter}

The second option to produce ham also works along with \hyperref[sec:farm]{Farms}, just like the \hyperref[sec:bread]{Bread System}. The \hyperref[sec:pigfarm]{Pig Farmer} grows one pig every 62 seconds, just like the \hyperref[sec:slaughterhouse]{Slaughter} kills a pig every 62 seconds. A \hyperref[sec:pigfarm]{Pig Farmer} needs wheat and water to feed his pigs, giving those right to the \hyperref[sec:slaughterhouse]{Slaughter}. Therefore the optimal ratio consists of 9 \hyperref[sec:farm]{Farms}, 7 \hyperref[sec:pigfarm]{Pig Farms}, 5 \hyperref[sec:well]{Wells} and 7 \hyperref[sec:slaughterhouse]{Slaughterhouses} to feed about 5,5 mines.

This renewable system uses up 16 tools in total.

\subsection{Food System Comparison}
\label{sec:foodsystemcomparison}

As the \hyperref[sec:fish]{Fish based system} (takes 4 tools per mine, non-renewable) and the \hyperref[sec:hamhunter]{Ham producing, Hunter-based system} (very short living, non-renewable) aren't well suited for a rich ore economy, this comparison mainly focuses on the renewable \hyperref[sec:bread]{Bread based-} and \hyperref[sec:hamslaughter]{Meat producing, Slaughter-based system}.

Here is a direct comparison of the optimal ratios of both systems in terms of \textbf{tool efficiency}, \textbf{production efficiency}, \textbf{amount of buildings} and \textbf{amount of spots} taken (This means, a Tier 1 building takes 1 spot, a Tier 2 building takes 2 spots and a Tier 3 building takes 3 spots. This is not an in-game mechanic, rather a way to summarize the Tier efficiency of a system), calculated down to feed a single mine of any type:

\begin{itemize}
  \item Bread requires 2,75 tools per mine while Ham takes up $2.\overline{90}$
  \item Bread feeds exactly 4 mines with a setup while Ham can feed about 5.5, adding unnecessary complexity
  \item Bread takes 4,75 buildings per mine while Ham requires $5.\overline{09}$
  \item Bread takes 10,25 spots per mine while Ham requires $12.\overline{18}$
\end{itemize}

As you can see, the \hyperref[sec:bread]{Bread system} wins every time and should therefore always be chosen over the \hyperref[sec:hamslaughter]{Meat-based alternative}.

\section{Ships and Harbor}
\label{sec:shipsandharbor}

\subsection{The Harbor}
\label{sec:harbor}

\textit{See also: \hyperref[sec:harbor_building]{Harbor (Building)}}\\

The \hyperref[sec:building_harbor]{Harbor} is the first step to discovering \hyperref[sec:islands]{Islands} in The Settlers II - 10th Anniversary. It acts just like every ordinary \hyperref[sec:stockhouse]{Stockhouse}, just that it has the ability to transfer goods and workers over to \hyperref[sec:ships]{Ships}. To send a \hyperref[sec:ships]{Ship} to discover a new \hyperref[sec:islands]{Island}, you must prepare a mission in the harbor's UI. This will get 4 \hyperref[sec:plank]{Planks}, 6 \hyperref[sec:stone]{Stones} and 1 \hyperref[sec:constructor]{Constructor} reserved into the harbor. If a \hyperref[sec:ships]{Ship} is ready, these things will get loaded unto the ship and you can select a direction where the ship should look for a place in the \hyperref[sec:ships]{ship's UI}.

\subsection{Ships}
\label{sec:ships}

\textit{See also: \hyperref[sec:shipyard]{Shipyard}}\\

Ships are made in the \hyperref[sec:shipyard]{Shipyard}. When ready for a mission, which was started previously in the \hyperref[sec:harbor]{Harbor} UI, you can select a direction where the ship should look for a Harbor when clicking on it. You can also cancel the Mission if there are no available Harbors.

\subsection{Goods Transport}
\label{sec:goodstransport}

When a building demands something, the nearest \hyperref[sec:stockhouse]{Stockhouse} is asked to bring the good over. If there is no Stockhouse with the required good reachable over land and the demanded good is just coming out of a factory, this will go to the demanding building. Otherwise, the game will check for the item on an \hyperref[sec:islands]{Island}. If there is one, the item will be brought to the Islands \hyperref[sec:harbor]{Harbor} and will then eventually be brought to the Island or Mainland requesting the item via a \hyperref[sec:ships]{Ship}. To ensure a flowing good exchange, there should be at least 2 or 3 Ships dedicated to each Island.

\section{Common Pitfalls}
\label{sec:commonpitfalls}

When there is so much strategy involved, there's also lots of room for error. Here you will learn of common pitfalls you need to avoid, some being more fatal than others.

\subsection{Beginner Struggles}
\label{sec:beginnerstruggles}

The following describes common issues and solutions I learned while being new to the game, as well as when introducing new players.

These will be ordered from early to later during a game.

\subsubsection{Wood Issues}
\label{sec:woodissues}

This is something even more experienced players can struggle with sometimes. At the beginning of a game, one of the most important resources is wood. You need it for every single building in the game, as well as for \hyperref[sec:ships]{Ships} and \hyperref[sec:toolsmith]{Tools}. Especially at low \hyperref[sec:startingresources]{starting resources}, wood is spare when starting a new game. While you can get infinite wood from a good \hyperref[sec:woodindustry]{wood industry}, it's not always easy to get one up. If not enough wood is produced at the beginning, or even no wood at all (e.g. when other building were prioritized so there's no wood left for \hyperref[sec:woodcutter]{woodcutters} and \hyperref[sec:sawmill]{sawmills}), nothing can be built anymore and you're stuck during the part of the game where time matters the most. To solve this issue, other buildings need to be destroyed so the remaining can be used to build up wood producers.

As this is something you definitely must avoid, you have to make sure to follow the \hyperref[sec:woodindustry]{wood industry section} before spending the valuable resource elsewhere.

\subsection{Intermediate Issues}
\label{sec:intermediateissues}

Now follows a collection of things intermediate to advanced players should look after to optimize their gameplay. Note that these tips are not necessarily needed to win a game against bots. However, they will greatly enhance stability of everything you do.

\subsubsection{Over-Expansion}
\label{sec:overexpansion}

Over-Expansion often happens when one party claims a lot of enemy territory in a short time by taking military buildings and connecting them only with simple roads. There won't be many buildings aside from military, making the \hyperref[sec:minimap]{minimap} look empty. However, this visual issue is not the main problem with this approach.

Especially on big maps, over-expansion will cause the following problems.

\begin{itemize}
    \item Long transportation roads, making buildings take much longer to build or upgrade
    \item Replenishment soldiers taking long to arrive after a battle
    \item Carelessly created roads don't fit into a well organized \hyperref[sec:roadsystem]{road system}
    \item Soldiers remain in buildings far from the front if those aren't emptied
    \item Makes leveling soldiers really hard when no \hyperref[sec:generalgenerator]{General Generator} is used
    \item Rapid expansion makes keeping track of natural resources (e.g. ore, fish) very hard
    \item Attacking too fast gets low-health \hyperref[sec:general]{generals} killed before they could heal
\end{itemize}

And possible more.

The most dangerous points are the long transport ways. Without a \hyperref[sec:decentralization]{decentralized} economy near the borders you have a big disadvantage against an opponent with a more compact boundary, or a better distributed industry, possibly getting overwhelmed, possible forcing you into a \hyperref[sec:completeerasure]{complete erasure}.

To avoid this risk, make sure you don't attack too rapidly (unless strategically appropriate, e.g. when destroying gold industry or \hyperref[sec:storehouse]{storehouses}) and instead care about each chunk of new territory as much as you would when discovering new land by \hyperref[sec:expanding]{expanding} regularly.

\subsection{Bugs}
\label{sec:bugs}

As The Settler II (10th anniversary) never really got any content updates, there are some minor, but gameplay-impacting bugs you should be aware of.

\subsubsection{Builders stuck on an Island}
\label{sec:buildersstuckonisland}

When using \hyperref[sec:shipsandharbor]{ships and the harbor} to bring your forces to an island, you will want to build up a decentralized industry at some point. Unfortunately, there's a little bug that could make \textbf{all} your builders and/or \hyperref[sec:leveler]{levelers} stuck on an island when requesting a new building, and no one on the island is ready to take the job. While normally, workers would go back to the mainland when needed there, in this case, they don't. This will completely freeze progress on your mainland and other islands. To fix this, build a \hyperref[sec:storehouse]{storehouse} and send the workers to it, out of the harbor. After all are in the storehouse, allow them to go back to the harbor and they will go back to the mainland when needed.

To generally prevent this from happening it's advisable to request a storehouse right when building a harbor.

\section{Bonus}
\label{sec:bonus}

\subsection{Bonus A: Challenges}
\label{sec:challenges}

\subsubsection{The Ultimate Challenge}
\label{sec:challenge_ultimate}

You really studied the game and want to show off your Skill using an incredibly hard challenge? This one's for you.

\begin{itemize}
  \item Map: Paradise Island
  \item \hyperref[sec:startingresources]{Starting Resources}: Medium
  \item Win Condition: Defeat all Enemies
  \item Fog of War: On
  \item Enemies: All in the same team, highest difficulty
\end{itemize}

This challenge is really hard because you need to defeat 5 Enemies all by yourself. Because this map is so huge (240*240 Spots) you cannot just rush your enemies. You need to play smart and for resources. Remember, they got five times as many as you did. Especially Iron will be tough, make sure you don't lose any of your soldiers before letting them become \hyperref[sec:general]{Generals}.

\subsubsection{The Ultimate Ultimate Challenge}
\label{sec:challenge_ultimate_ultimate}

You beat the last challenge? Good job! But there's more. This time you got even harder goals:

\begin{itemize}
  \item Map: Paradise Island
  \item \hyperref[sec:startingresources]{Starting Resources}: Low
  \item Win Condition: None (Endless Game)
  \item Fog of War: On
  \item Enemies: All in the same team, highest difficulty
  \item HQ's: All randomized
  \item Custom Goal: Completely destroy every Enemy, make sure no one's got any territory left
\end{itemize}

As you can see, not only have your \hyperref[sec:startingresources]{Starting Resources} halved, the Headquarters are also randomized and, most importantly, you can't just chill on the main island erasing your opponents. You must dominate on every single island as well (there are 7 in total). However, as soon as you got your fleet ready to take the outer (and inner) land, you should never ever run out of resources again.

\subsubsection{More}
\label{sec:challenge_more}

Surely those challenges were hard. However, they're not impossible! You can always look for custom maps, or even make your own one! Playing with other experts can also be quite challenging. You found a really, really (and I mean really) hard or even impossible challenge? Why not share it by \hyperref[sec:contributing]{Contributing}.

\end{document}

